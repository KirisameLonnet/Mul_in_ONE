\documentclass[UTF8]{ctexart}
\usepackage{geometry}
\usepackage{setspace}
\usepackage{hyperref}
\geometry{a4paper, margin=2.5cm}
\setstretch{1.25}

\title{2025年上海大学人工智能创新大赛:作品研究报告\\\Large 多智能体协同RAG增强角色扮演系统(Multi-in-One,Mio)}
\author{申报者:冯思源\\所在学院:机电工程与自动化学院}
\date{\today}

\begin{document}
\maketitle
\tableofcontents
\newpage

\section{引言}
\subsection{项目研究的背景及意义}
\subsection{项目研究主要内容与创新点}

\section{Mio-多智能体协同RAG增强角色扮演平台系统设计}
\subsection{Mio平台整体设计方案}
\subsubsection{系统设计目标与约束原则}
\subsubsection{系统总体逻辑架构}
\subsubsection{核心业务流转机制}
\subsection{Mio平台核心架构设计}
\subsubsection{多智能体协同架构}
\subsubsection{RAG检索架构}
\subsection{Mio平台软件架构设计}
\subsubsection{后端技术栈与服务实现}
\subsubsection{前端技术栈与交互实现}
\subsubsection{结构化数据存储}
\subsubsection{SaaS多租户与数据库配置架构}
\subsection{Mio平台开发环境设计与搭建}
\subsubsection{基于 Nix 的确定性环境构建}
\subsubsection{后端依赖管理与虚拟环境 (uv)}
\subsubsection{前端依赖管理与工程化构建 (npm + Vite)}

\section{Mio系统实验结果与分析}
\subsection{xxxx平台实验结果}
\subsection{xxxx平台实验结果分析}

\section{xxxx系统设计总结与展望}
\subsection{xxxx平台设计总结}
\subsection{xxxx平台设计展望}

\appendix
\section{附录}

\end{document}
